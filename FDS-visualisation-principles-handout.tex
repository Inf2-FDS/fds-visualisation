\documentclass[10pt]{article}
\usepackage{times}
\usepackage[top=0.4in,left=0.6in,right=0.6in,bottom=0.3in,landscape,a4paper]{geometry}
% \usepackage[handout]{fds}
% \todosfalse
\usepackage[math]{iwona}
\usepackage{natbib}
\usepackage{hyperref}
\usepackage{multicol}
\usepackage{enumitem}
\newcommand{\firstterm}[1]{\textbf{#1}}
%\renewcommand{\paragraph}[1]{\hrule\hrule\hrule\vspace*{2pt}\noindent\textbf{#1} \vspace{2pt}\hrule\vspace{1pt}\noindent}
\renewcommand\bibsection{\paragraph{\refname}}
\setlength{\parindent}{0pt}

\usepackage{titlesec}

\titleformat{\paragraph}
{\titlerule[1pt]%
\vspace{.2ex}%
\normalfont\bfseries}
{\thesection.}{.2em}{}[\hrule\vspace*{-1ex}]

\begin{document}

\raggedright
\pagestyle{empty}
% \maketitle

\begin{center}
  \large\textbf{Informatics 2 -- Foundations of Data Science: Visualisation
    principles and guidance}
\end{center}

\setlist[itemize]{leftmargin=10pt,noitemsep,parsep=0pt}

\setlength{\columnseprule}{.2pt}
\begin{multicols}{3}
\paragraph{Principle 1: Show the data} 

Aim to show as much of the data as possible without leading to a
confusing visualisation. There are often multiple ways of representing
the same dataset, and no ``right'' answer. The following guidance
% on arranging the the graphical elements of the plot
should help show as much of the data as possible:
\begin{itemize}
\item \textbf{Choose an appropriate plot type.} Basic types include:
  \begin{itemize}
  \item Bar charts: for plotting numeric variables associated with
    categorical or ordinal variables, e.g.~the mean weight
    (numeric variable) of male and female (categorical variable)
    squirrels.
  \item Line charts: for showing trends of numerical variables over
    time (a numerical variable).
  \item Scatter plots: show the relationship between two numeric
    variables.
  \item Box plots: represent the distribution of a numeric variable for
    multiple categories, e.g.~the weights of male and female
    squirrels.
  \item Histograms and density plots: good for showing the
    distribution of a single variable.
  \end{itemize}
\item \textbf{Show multiple variables by using length, shape, size and colour:}
  \begin{itemize}
  \item Use shape and colour to create extra dimensions for
    categorical variables. E.g.~in a scatter plot of squirrel weight
    versus length, indicate sex using colour, thus displaying 3
    variables. In addition, indicate age categories (ordinal) by
    changing the size or the shape of the markers (4 variables).
    But take care that the plot is not too complex to read.
    % However, adding information using marker properties can detract
    % from the plot.
  \item Bar charts can be extended to two categorical variables and one
    numerical variable by using colour.
  \end{itemize}
\item \textbf{Use colour effectively.} (\citealp{WexlEtal17big}, pp.~14--18)
  \begin{itemize}
  \item Choose an appropriate colour scale, depending on if the data
    is sequential (numeric), diverging (numeric with a zero point in
    the the scale) or categorical.
  \item Colour can also be used to highlight features in the plot,
    e.g.~the largest two bars in a bar plot.
  \end{itemize}
\item \textbf{Encourage the eye to compare several pieces of data}, e.g.~by
  using multiple plots with the same scale.
  \begin{itemize}
  \item \citet{WexlEtal17big}, p.~31, is a nice example of how this
    can work better than using multiple symbols on a plot (p.~30).
  \end{itemize}

\item \textbf{Present many numbers in a small space}
  \begin{itemize}
  \item A box plot takes up as much space as a barplot, but conveys
    more information. For example, a box plot of the squirrel's weight
    versus sex shows information about the distribution of the weight
    as well as the median weight.
  \end{itemize}
\item \textbf{Choose appropriate transforms}
  \begin{itemize}
  \item Transforming data can make features of it clearer. For
    example, plotting the value of Bitcoin over time shows very little
    detail about the early history of the currency, when it was not
    valuable. However, plotting the log of the value of Bitcoin on the
    $y$-axis allows this detail to be seen.
  \end{itemize}
\end{itemize}


\paragraph{Principle 2:  Make the meaning of the data clear}

A visualisation is meaningless if it's not labelled. Every plot should have:
\begin{itemize}
\item Title or caption
\item Axis labels as English words
\item Units given, where appropriate (e.g.~``Length (mm)''
  \emph{not just} ``Length'')
\item All variables labelled -- e.g.~a legend indicating the colours
  used to represent squirrel sex
\item Graphical and textual annotation where appropriate -- e.g.~it
  can be helpful to highlight a time series with events that you know
  about
\end{itemize}

\paragraph{Principle 3: Avoid distorting what the data have to say}

Choices in visualisation design can lead to the instant impression
given by preattentive processing of the visualisation being quite
different to the numbers in the dataset. \citet{Tuft82visu1ed} 
measures the level of distortion in a visualisation by the ``Lie
factor'':
\begin{displaymath}
  \textrm{Lie factor} = \frac{\textrm{size of effect shown in graphic}}{\textrm{size of
  effect in data}}
\end{displaymath}
The following guidelines help to avoid distorting the data:
\begin{itemize}
\item \textbf{Use appropriate scales and baselines}
  \begin{itemize}
  \item A very common problem is that the baseline (i.e.~the lowest
    point on the $y-$axis) in a bar chart is not zero. This can lead to
    small differences appearing large.
  \end{itemize}
\item \textbf{Be aware of limitations of our perception of size}
  \begin{itemize}
  \item Although marker area can be useful for indicating categories,
    humans are not very good at relating the area to a quantity -- we
    are much better at comparing lengths.
  \end{itemize}
\end{itemize}


\paragraph{Principle 4:  Make the data accessible}

A visualisation is meaningless if it's illegible and loses impact if
it's difficult to read. To ensure data is accessible:
\begin{itemize}
\item \textbf{Make sure text is legible}, i.e.~font size of minimum 8~points in
  a PDF, or about 20 points in a presentation. (It is surprising how
  often talks are given in which it's impossible to read the labels on
  plots even from the front row.)
\item \textbf{Use colours that work for people with colour-vision deficiency.}
  \citet{WexlEtal17big}, Chapter 1 has an excellent introduction to
  using colour in visualisations.
\end{itemize}


\paragraph{Principle 5:  Focus on the content}

Give the viewer's brain as little work to do as possible.
\begin{itemize}
\item Avoid chartjunk -- e.g.~colours that don't have any meaning.
\item Reduce clutter
\item Use consistent colours for plots in the same study
\item Use an appropriate number of decimal places
\item Check spelling is correct
\end{itemize}

\bibliographystyle{apalike}
\bibliography{main}
\end{multicols}

\vskip0.1em

\begin{center}
  \footnotesize Version~1.0 \copyright 2022-2024 by David Sterratt,
  licensed under CC BY-SA 4.0
  Source: \url{https://github.com/Inf2-FDS/fds-visualisation}
\end{center}
\end{document}

%%% Local Variables:
%%% mode: latex
%%% End:

% LocalWords:  Wexler et al EDA pre Tukey WexlEtal Anscombe th visu
% LocalWords:  Playfair Schw bett chartjunk Tufte itemize leftmargin
% LocalWords:  noitemsep parsep
