\documentclass{article}

\usepackage[width=7in,height=10in,top=1in,bottom=1in]{geometry}
\usepackage{times}

\begin{document}

\begin{tabular}{@{}|r|p{6.5in}|@{}}
  \multicolumn{2}{c}{\textbf{Overall effectiveness of visualisation}} 
  \\
  \hline
  12 
  &An outstanding exemplar of the application of all five visualisation principles, where the combination of design choices makes an exceionally effective visualisation that makes the answer to the question ``pop out'' and displays a lot of data so that further questions are provoked. Worthy of publication in the \emph{Financial Times}.
  \\
  \hline10 
  &A good exemplar of the application of all five visualisation principles, where the combination of design choices makes a very effective visualisation that makes the answer to the question clear. Worthy of publication in an academic journal.
  \\
  \hline
  8 
  &The visualisation principles of making the data accessible, making
  the meaning clear (title, axis and variable labels) and not distorting
  the data are applied and there is good evidence of application of the
  principles of showing the data and focusing on the content. The design choices work together to make an effective visualisation that makes the answer to the question clear.
  \\
  \hline6 
  &There is evidence of applying the visualisation principles, but one of the principles of making the data accessible, making the data clear or avoiding distorting the data may not be fully applied.  Nevertheless, the design choices work together to make a visualisation that is pertinent to the question, and addresses it reasonably clearly.
  \\
  \hline4 
  &There is some evidence of applying the visualisation principles, but two of the principles of making the data accessible, making the data clear or avoiding distorting the data may be violated. The design choices work together to make a visualisation that requires effort to relate to the question, and may not address it clearly.
  \\
  \hline2 
  &There is little evidence of applying the visualisation principles. The design choices work together to make a visualisation that requires substantial effort to relate to the question, but does not address it clearly.
  \\
  \hline0 
  &The visualisation requires a great deal of effort to understand and
  relate to the question
  \\
  \hline
  \multicolumn{2}{c}{\textbf{Meaning of data: title}}\\
  \hline
  1 & An informative and perfect title is provided \\
  \hline
  $\frac{1}{2}$ & Title present but could be more informative \\
  \hline
  0 & No title
  \\
  \hline
  \multicolumn{2}{c}{\textbf{Meaning of data: labels}}\\
  \hline
  1 
  &Variables labelled clearly with units
  \\
  \hline $\frac{1}{2}$ 
  &Variables labelled  but in some cases the labels/titles are not clear or accurate
  \\
  \hline
  0 
  &Meaning of data not clear
  \\
  \hline
  \multicolumn{2}{c}{\textbf{Data not distorted}}
  \\
  \hline
  1 
  & Scales and baselines and marker sizes all appropriate
  \\
  \hline $\frac{3}{4}$ 
  &Scale is fine but the plot is confusing due to bad ordering of x axis values
  \\
  \hline $\frac{1}{2}$ 
  & The scale is not appropriate and/or could have incorporated a baseline.
  \\
  \hline $\frac{1}{2}$ 
  &Multiple scales in the different subplots could distort the data
  \\
  \hline $\frac{1}{2}$ 
  &The effect size of the visualisation is not equal to effect size of the data
  \\
  \hline 0 
  &Scale makes the plot confusing.
  \\
  \hline
  \multicolumn{2}{c}{\textbf{Accessibility: Text size}}
  \\
  \hline
  1 
  &All text has adequate font size, easy to read
  \\
  \hline $\frac{1}{2}$ 
  &Some text is too small, but just legible
  \\
  \hline0 
  &Some text is illegible at a reasonable magnification - i.e. much smaller than the size of the body text
  \\
  \hline
  \multicolumn{2}{c}{\textbf{Accessibility: colour}}
  \\
  \hline
  1 
  & Colours are colour-vision-deficient-friendly
  \\
  \hline $\frac{1}{2}$ 
  & Colours are colour-vision-deficient-friendly but the colour used makes the plot confusing
  \\
  \hline $\frac{1}{2}$ 
  & Colours are not colour-vision-deficient-friendly but
  numbers/labels make the plot accessible for people with colour
  vision deficiency
  \\
  \hline $\frac{1}{2}$ 
  & Colours are colour-vision-deficient-friendly but
  yellow/orange/blue on white background is used
  \\
  \hline
  0 
  &  Colours are not colour-vision-deficient-friendly
  \\
  \hline
  \multicolumn{2}{c}{\textbf{Plot size and focus on the content}}
  \\
  \hline
  1 
  & Plot is adequately sized and does not contain chartjunk or clutter
  \\
  \hline $\frac{1}{2}$ 
  &Plot is not adequately sized or contains chart junk or clutter
  \\
  \hline0 
  &Plot does not fit on one page or doesn't focus on the content
  \\
  \hline
\end{tabular}

\end{document}


