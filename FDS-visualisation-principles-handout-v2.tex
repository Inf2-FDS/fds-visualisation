\documentclass[10pt]{article}
\usepackage{times}
\usepackage[top=0.3in,left=0.3in,right=0.3in,bottom=0.5in,landscape,a4paper,
footskip=0.2in]{geometry}
% \usepackage[handout]{fds}
% \todosfalse
\usepackage[math]{iwona}
\usepackage{natbib}
\usepackage{hyperref}
\urlstyle{same}
\usepackage{multicol}
\usepackage{enumitem}
\setlist{topsep=0pt}
\newcommand{\firstterm}[1]{\textbf{#1}}
%\renewcommand{\paragraph}[1]{\hrule\hrule\hrule\vspace*{2pt}\noindent\textbf{#1} \vspace{2pt}\hrule\vspace{1pt}\noindent}

\usepackage{titlesec}
\usepackage{fancyhdr}

\titleformat{\paragraph}
{\titlerule[1pt]%
\vspace{.2ex}%
\normalfont\bfseries}
{\thesection.}{.2em}{}[\hrule\vspace*{-1ex}]


% For callout boxes 
\usepackage[skins,breakable]{tcolorbox}
% \@ifpackageloaded{tcolorbox}{}{\usepackage[skins,breakable]{tcolorbox}}
\usepackage{fontawesome5}

% \newcommand{\principle}[2]{\paragraph{#1} #2 \\[0.2em]\hrule \\[0.2em]}

\definecolor{quarto-callout-tip-color}{HTML}{00A047}
\definecolor{quarto-callout-tip-color-frame}{HTML}{02b875}
\definecolor{quarto-callout-standard-color}{HTML}{909090}
\definecolor{quarto-callout-standard-color-frame}{HTML}{acacac}

\newlength{\calloutparindent}
\newenvironment{callout}[2][standard]{
  \setlength{\calloutparindent}{\parindent}
  \begin{tcolorbox}[enhanced jigsaw,
    leftrule=.15mm, toprule=.15mm,
    colframe=quarto-callout-#1-color-frame, opacityback=0,
    colback=white, coltitle=black, rightrule=.15mm, breakable,
    colbacktitle=quarto-callout-#1-color!10!white,
    opacitybacktitle=0.6, titlerule=0mm, left=1mm,right=1mm,
    title=\textcolor{quarto-callout-#1-color}{\expandafter\csname
      quartocallout#1icon\endcsname}{\textbf{#2}},
    bottomrule=.15mm, toptitle=1mm, arc=.35mm, bottomtitle=1mm,
    before=\par\medskip]%
    \setlength{\parindent}{\calloutparindent}\noindent\ignorespaces}%
  {\end{tcolorbox}}

\newcommand{\principle}[3]{\begin{callout}{#1}\raggedright#2\tcblower\raggedright#3\end{callout}}

%\renewcommand\bibsection{\paragraph{\refname}}
%\renewcommand\bibsection{\begin{callout}{\refname}\end{callout}}
\renewcommand\bibsection{\vspace{-1.3em}}
\setlength{\parindent}{0pt}


\begin{document}
\fancyfoot[C]{
  \footnotesize Version 2.0-beta \copyright 2022-2025 by David Sterratt, licensed under CC
  BY-SA 4.0
  Source: \url{https://github.com/Inf2-FDS/fds-visualisation}}
\thispagestyle{fancy}
\raggedright
\pagestyle{empty}
% \maketitle

\begin{center}
  \large\textbf{Informatics 2 -- Foundations of Data Science: Visualisation
    principles and guidance}
\end{center}

\setlist[itemize]{leftmargin=10pt,noitemsep,parsep=0pt}

% \setlength{\columnseprule}{.2pt}
\begin{multicols}{3}
  \principle{Principle 1: Show the data}%
  {Show as much of the data as possible without making a confusing
    visualisation. There are multiple ways of representing the same
    dataset, and no ``right'' answer.}%
{\begin{itemize}[itemsep=1ex]
\item \textbf{Identify the variables and their type.} For tabular data
  in a tidy (long) format, each column corresponds to a numeric,
  categorical, ordinal or unstructured variable.
\item
  \textbf{Choose an appropriate plot type to show one or two
    variables.}
  Use the guide at \url{https://www.data-to-viz.com/}. E.g.:
  \begin{itemize}[topsep=0pt]
  \item One numeric variable $\rightarrow$ histogram or density plot
    shows distribution
  \item One categorical variable $\rightarrow$ bar plot
    shows counts
  \item One categorical variable and one numeric variable
    $\rightarrow$ bar plot can show mean of numeric variable for each
    category; box plot or violin plot show distribution.
  \item Two unordered numeric variables $\rightarrow$ scatter plot
    shows relationship
  \item One ordered and one ordered numeric variable (e.g.~time series)
    $\rightarrow$ line plot
  \end{itemize}
\item
  \textbf{Consider showing extra variables by using length, shape,
    size and colour.}
  \begin{itemize}
  \item E.g.~In a scatter plot (two numeric variables), the colour and
    shape of each marker can represent two categorical variables, thus
    displaying four variables. Size can represent ordinal variables.
  \item But assess whether the plot is too complex to read.
    % However, adding information using marker properties can detract
    % from the plot.
  % \item Bar charts can be extended to two categorical variables and one
  %   numerical variable by using colour.
  \end{itemize}
\item \textbf{Consider using a table.} Data patterns are clearer in
  tables than graphics. However, tables are a form of visualisation,
  and good for conveying raw data or dealing with large numbers of
  variables.
\item \textbf{Use colour effectively.} (\citealp{WexlEtal17big}, pp.~14--18)
  \begin{itemize}
  \item Choose an appropriate colour scale, depending on if the data
    is sequential (numeric), diverging (numeric with a zero point in
    the the scale) or categorical.
  \item Colour can also be used to highlight features in the visualisation,
    e.g.~the largest two bars in a bar plot or the largest values in
    each column of a table.
  \end{itemize}
\item \textbf{Encourage the eye to compare several pieces of data}
  \begin{itemize}
  \item E.g.~use multiple plots with the same scale (``small
    multiples''), which can work better than using large numbers of
    symbols or colours on a single plot
  \end{itemize}
\item \textbf{Present many numbers in a small space}
  \begin{itemize}
  \item E.g.~A box plot of a numeric variable uses as much space as a
    bar plot, but conveys more information.
  \end{itemize}
\item \textbf{Choose appropriate transforms}
  \begin{itemize}
  \item E.g.~for a positive variable that varies over many orders of
    magnitude (e.g.~the value of Bitcoin) a log transform can show
    changes when the variable is both small and large.
  \end{itemize}
\end{itemize}
}

\principle{Principle 2:  Make the meaning of the data clear}%
{A visualisation is meaningless if it's not labelled.}%
{Every plot should have:
\begin{itemize}
\item Title or caption
\item Axis labels as English words
\item Units given, where appropriate (e.g.~``Length (mm)''
  \emph{not just} ``Length'')
\item All variables labelled on axes or legends
\item Graphical and textual annotation where appropriate --
  e.g.~highlight a time series with known events
\item Description of any error bars, e.g.~95\% confidence interval,
  standard deviation, or standard error
\end{itemize}
}

\principle{Principle 3: Avoid distorting what the data have to say}%
{Design choices can mean the instant impression given by preattentive
processing of the visualisation is quite different to the numbers in
the dataset.}%
{
% \citet{Tuft82visu1ed} 
% measures the level of distortion in a visualisation by the ``Lie
% factor'':
% \begin{displaymath}
%   \textrm{Lie factor} = \frac{\textrm{size of effect shown in graphic}}{\textrm{size of
%   effect in data}}
% \end{displaymath}
\begin{itemize}[itemsep=1ex]
\item \textbf{Use appropriate scales and baselines}
  \begin{itemize}
  \item A common problem is that the baseline (i.e.~the lowest point
    on the $y-$axis) in a bar chart is not zero, leading to small
    differences appearing large.
  \end{itemize}
\item \textbf{Be aware of limitations of human perception}
  \begin{itemize}
  \item Humans are better at comparing lengths than areas
    $\rightarrow$ consider whether area represents a given numeric
    variable well
  \item Humans are better at comparing lengths than angles
    $\rightarrow$ consider alternatives to pie charts, especially with
    many categories
  \end{itemize}
\end{itemize}
}

\principle{Principle 4: Make the data accessible}%
{Make visualisations accessible so that they are meaningful for
  everyone.}%
{
\begin{itemize}[itemsep=1ex]
\item \textbf{Make sure text is legible}, i.e.~font size of minimum
  8~points in a PDF, or about 20 points in a presentation.
  (Surprisingly often in talks, it's impossible to read plot labels,
  even from the front row.)
\item \textbf{Use colours that work for people with colour-vision deficiency.}
  \citet{WexlEtal17big}, Chapter 1 has an excellent introduction to
  using colour in visualisations.
\end{itemize}
}

\principle{Principle 5: Focus on the content}%
{Minimise distractions for the viewer's brain.}%
{\begin{itemize}
\item Avoid chartjunk -- e.g.~colours that don't have any meaning, or
  3D bar charts
\item Reduce clutter -- e.g.~vertical grid lines with a categorical \emph{x}-axis
\item Use consistent colours for plots in the same study
\item Use an appropriate number of decimal places
\item Check spelling is correct
\end{itemize}
}

\bibliographystyle{apalike}
\begin{callout}{Reference}
\bibliography{main}  
\end{callout}

\end{multicols}
\end{document}

%%% Local Variables:
%%% mode: latex
%%% End:

% LocalWords:  Wexler et al EDA pre Tukey WexlEtal Anscombe th visu
% LocalWords:  Playfair Schw bett chartjunk Tufte itemize leftmargin
% LocalWords:  noitemsep parsep
